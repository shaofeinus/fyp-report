% ******************************************************************************
% ****************************** Custom Margin *********************************

% Add `custommargin' in the document class options to use this section
% Set {innerside margin / outerside margin / topmargin / bottom margin}  and
% other page dimensions
\ifsetCustomMargin
  \RequirePackage[left=37mm,right=30mm,top=35mm,bottom=30mm]{geometry}
  \setFancyHdr % To apply fancy header after geometry package is loaded
\fi

% Add spaces between paragraphs
%\setlength{\parskip}{0.5em}
% Ragged bottom avoids extra whitespaces between paragraphs
\raggedbottom
% To remove the excess top spacing for enumeration, list and description
%\usepackage{enumitem}
%\setlist[enumerate,itemize,description]{topsep=0em}

% *****************************************************************************
% ******************* Fonts (like different typewriter fonts etc.)*************

% Add `customfont' in the document class option to use this section

\ifsetCustomFont
  % Set your custom font here and use `customfont' in options. Leave empty to
  % load computer modern font (default LaTeX font).
  %\RequirePackage{helvet}

  % For use with XeLaTeX
  %  \setmainfont[
  %    Path              = ./libertine/opentype/,
  %    Extension         = .otf,
  %    UprightFont = LinLibertine_R,
  %    BoldFont = LinLibertine_RZ, % Linux Libertine O Regular Semibold
  %    ItalicFont = LinLibertine_RI,
  %    BoldItalicFont = LinLibertine_RZI, % Linux Libertine O Regular Semibold Italic
  %  ]
  %  {libertine}
  %  % load font from system font
  %  \newfontfamily\libertinesystemfont{Linux Libertine O}
\fi

% *****************************************************************************
% **************************** Custom Packages ********************************

% ************************* Algorithms and Pseudocode **************************

%\usepackage{algpseudocode}


% ********************Captions and Hyperreferencing / URL **********************

% Captions: This makes captions of figures use a boldfaced small font.
%\RequirePackage[small,bf]{caption}

\RequirePackage[labelsep=space,tableposition=top]{caption}
\renewcommand{\figurename}{Fig.} %to support older versions of captions.sty


% *************************** Graphics and figures *****************************

%\usepackage{rotating}
%\usepackage{wrapfig}

% Uncomment the following two lines to force Latex to place the figure.
% Use [H] when including graphics. Note 'H' instead of 'h'
%\usepackage{float}
%\restylefloat{figure}

% Subcaption package is also available in the sty folder you can use that by
% uncommenting the following line
% This is for people stuck with older versions of texlive
%\usepackage{sty/caption/subcaption}
\usepackage{subcaption}

% ********************************** Tables ************************************
\usepackage{booktabs} % For professional looking tables
\usepackage{multirow}

%\usepackage{multicol}
%\usepackage{longtable}
%\usepackage{tabularx}


% *********************************** SI Units *********************************
\usepackage{siunitx} % use this package module for SI units


% ******************************* Line Spacing *********************************

% Choose linespacing as appropriate. Default is one-half line spacing as per the
% University guidelines

% \doublespacing
% \onehalfspacing
% \singlespacing


% ************************ Formatting / Footnote *******************************

% Don't break enumeration (etc.) across pages in an ugly manner (default 10000)
%\clubpenalty=500
%\widowpenalty=500

%\usepackage[perpage]{footmisc} %Range of footnote options


% *****************************************************************************
% *************************** Bibliography  and References ********************

%\usepackage{cleveref} %Referencing without need to explicitly state fig /table

% Add `custombib' in the document class option to use this section
\ifuseCustomBib
   \RequirePackage[square, sort, numbers, authoryear]{natbib} % CustomBib

% If you would like to use biblatex for your reference management, as opposed to the default `natbibpackage` pass the option `custombib` in the document class. Comment out the previous line to make sure you don't load the natbib package. Uncomment the following lines and specify the location of references.bib file

%\RequirePackage[backend=biber, style=numeric-comp, citestyle=numeric, sorting=nty, natbib=true]{biblatex}
%\bibliography{References/references} %Location of references.bib only for biblatex

\fi

% changes the default name `Bibliography` -> `References'
\renewcommand{\bibname}{References}


% ******************************** Roman Pages *********************************
% The romanpages environment set the page numbering to lowercase roman one
% for the contents and figures lists. It also resets
% page-numbering for the remainder of the dissertation (arabic, starting at 1).

\newenvironment{romanpages}{
  \setcounter{page}{1}
  \renewcommand{\thepage}{\roman{page}}}
{\newpage\renewcommand{\thepage}{\arabic{page}}}


% ******************************************************************************
% ************************* User Defined Commands ******************************
% ******************************************************************************

% *********** To change the name of Table of Contents / LOF and LOT ************

%\renewcommand{\contentsname}{My Table of Contents}
%\renewcommand{\listfigurename}{My List of Figures}
%\renewcommand{\listtablename}{My List of Tables}


% ********************** TOC depth and numbering depth *************************

\setcounter{secnumdepth}{2}
\setcounter{tocdepth}{2}


% ******************************* Nomenclature *********************************

% To change the name of the Nomenclature section, uncomment the following line

%\renewcommand{\nomname}{Symbols}


% ********************************* Appendix ***********************************

% The default value of both \appendixtocname and \appendixpagename is `Appendices'. These names can all be changed via:

%\renewcommand{\appendixtocname}{List of appendices}
%\renewcommand{\appendixname}{Appndx}

% *********************** Configure Draft Mode **********************************

% Uncomment to disable figures in `draftmode'
%\setkeys{Gin}{draft=true}  % set draft to false to enable figures in `draft'

% These options are active only during the draft mode
% Default text is "Draft"
%\SetDraftText{DRAFT}

% Default Watermark location is top. Location (top/bottom)
%\SetDraftWMPosition{bottom}

% Draft Version - default is v1.0
%\SetDraftVersion{v1.1}

% Draft Text grayscale value (should be between 0-black and 1-white)
% Default value is 0.75
%\SetDraftGrayScale{0.8}


% ******************************** Todo Notes **********************************
%% Uncomment the following lines to have todonotes.

%\ifsetDraft
%	\usepackage[colorinlistoftodos]{todonotes}
%	\newcommand{\mynote}[1]{\todo[author=kks32,size=\small,inline,color=green!40]{#1}}
%\else
%	\newcommand{\mynote}[1]{}
%	\newcommand{\listoftodos}{}
%\fi

% Example todo: \mynote{Hey! I have a note}


\usepackage{algorithm}
\usepackage[noend]{algpseudocode}

\makeatletter
\def\BState{\State\hskip-\ALG@thistlm}
\makeatother

\usepackage{graphicx, ifpdf, array, xstring, amsfonts, amsmath,multirow,rotating, nameref}
\newcommand{\currHeading}{@currentlabelname}

\newcommand{\kwTransaction}[2]{\textit{#1transaction#2}}
\newcommand{\kwBlock}[1]{\textit{block#1}}
\newcommand{\kwInput}[1]{\textit{input#1}}
\newcommand{\kwOutput}[1]{\textit{output#1}}
\newcommand{\kwCoin}[1]{\textit{coin#1}}
%\newcommand{\mathIn}[1]{$#1$}
%\newcommand{\varCoin}{c}
%\newcommand{\varSN}{S}
%\newcommand{\varCoinR}{r}

\newcommand{\varPComm}{p_{comm}}
\newcommand{\varQComm}{q_{comm}}
\newcommand{\varGComm}{g_{comm}}
\newcommand{\varHComm}{h_{comm}}
\newcommand{\varChallenge}{\mathcal{C}}
\newcommand{\varAccPoKCoinComm}{\mathfrak{C}_c}
\newcommand{\varAccPoKProofComm}[1]{C_#1}
\newcommand{\varAccPoKPComm}{\mathfrak{p}}
\newcommand{\varAccPoKQComm}{\mathfrak{q}}
\newcommand{\varAccPoKGComm}{\mathfrak{g}}
\newcommand{\varAccPoKHComm}{\mathfrak{h}}
\newcommand{\varAccPoKPProof}{p_{QRN}}
\newcommand{\varAccPoKQProof}{q_{QRN}}
\newcommand{\varAccPoKGProof}{g_{QRN}}
\newcommand{\varAccPoKHProof}{h_{QRN}}
\newcommand{\varPSok}{p_{SoK}}
\newcommand{\varQSok}{q_{SoK}}
\newcommand{\varGSok}{g_{SoK}}
\newcommand{\varHSok}{h_{SoK}}
\newcommand{\varLambdaSec}{\lambda_{zkp}}

\newcommand{\eqnAccum}[2]{Accumulate((N,\allowbreak#1),\allowbreak#2)\rightarrow A}
\newcommand{\eqnGenWit}[2]{GenWtiness((N,u),#1,C)\rightarrow #2}
\newcommand{\eqnAccVer}[1]{AccVerify((N,\allowbreak u),A,#1,w)\rightarrow \{0,1\}}
\newcommand{\eqnAccPoK}{NIZKPoK\{(c,w):AccVerify((N,u),A,c,w)=1 \wedge c \in [\mathfrak{A},\mathfrak{B}]\}}
\newcommand{\eqnAccPoKActual}{
	
	\[ NIZKPoK(c,\beta,\gamma,\delta,\varepsilon,\zeta,\varphi,\psi,\eta,\sigma,\xi): \{ \]
	
	\[ \varAccPoKCoinComm=\varAccPoKGComm^c \varAccPoKHComm^\varphi \wedge
	\varAccPoKGComm=(\varAccPoKCoinComm/\varAccPoKGComm)^\gamma \varAccPoKHComm^\psi \wedge 
	\varAccPoKGComm=(\varAccPoKGComm\varAccPoKCoinComm)^\sigma \varAccPoKHComm^\xi \wedge 
	\]
	
	\[
	\varAccPoKProofComm{r}=\varAccPoKGProof^\varepsilon \varAccPoKHProof^\zeta \wedge 
	\varAccPoKProofComm{c}=\varAccPoKGProof^c \varAccPoKHProof^\eta \wedge 
	A=\varAccPoKProofComm{w}^c (1/\varAccPoKHProof)^\beta \wedge 
	\]
	
	\[
	1=\varAccPoKProofComm{r}^c (1/\varAccPoKHProof)^\delta (1/\varAccPoKGProof)^\beta \wedge
	c \in [\mathfrak{A},\mathfrak{B}]\}
	\]
}
\newcommand{\eqnSNSoK}{ZKSoK[m]\{(c,r):c=\varGComm^S\varHComm^r\}}
\newcommand{\eqnSNSoKActual}{ZKSoK[m]\{(c,r,z):y=\expSoKCoinComm{\expPedCommCoin}\}}
\newcommand{\eqnCommPoK}{NIZKPoK\{(c,\varphi,z):\varAccPoKCoinComm=\varAccPoKGComm^{c}\varAccPoKHComm^{\varphi} \wedge y=\expSoKCoinComm{c}\}}

\newcommand{\expIntGroup}[1]{\mathbb{Z}_{#1}}
\newcommand{\expMulGroup}[1]{\mathbb{Z}_{#1}^{*}}
\newcommand{\expPedCommCoin}{\varGComm^{S}\varHComm^{r}}
\newcommand{\expSoKCoinComm}[1]{\varGSok^{#1}\varHSok^{z}}
\newcommand{\expOneToN}[2]{#1_1,\cdots,#1_#2}
\newcommand{\expAFrak}{\mathfrak{A}=-\varPComm2^{k'+k''+2}}
\newcommand{\expBFrak}{\mathfrak{B}=\varPComm2^{k'+k''+2}}
\newcommand{\expConcatAccPoK}{
	\varAccPoKCoinComm \|
	\varAccPoKProofComm{c} \|
	\varAccPoKProofComm{w} \|
	\varAccPoKProofComm{r} \|
	\varAccPoKGComm \|
	\varAccPoKHComm \|
	\varAccPoKGProof \|
	\varAccPoKHProof \|
	t_i \cdots
}
\newcommand{\expConcatSNSoK}{
	y \|
	S \|
	\varGSok \|
	\varHSok \|
	t_1 \|
	\cdots \|
	t_{\varLambdaSec} \|
	m
}
\newcommand{\expConcatPrimeSNSoK}{
	y \|
	S \|
	\varGSok \|
	\varHSok \|
	t'_1 \|
	\cdots \|
	t'_{\varLambdaSec} \|
	m
}
\newcommand{\expConcatCommPoK}{
	\varAccPoKCoinComm \|
	y \|
	t_1 \|
	t_2 \|
	\varAccPoKGComm \|
	\varAccPoKHComm \|
	\varGSok \|
	\varHSok
}
\newcommand{\expConcatSNSoKImproved}{
	y \|
	S \|
	\varGSok \|
	\varHSok \|
	t \|
	m
}
\newcommand{\expConcatAccPoKFull}{
	\varAccPoKCoinComm \| \allowbreak
	\varAccPoKProofComm{c} \| \allowbreak
	\varAccPoKProofComm{w} \| \allowbreak
	\varAccPoKProofComm{r} \| \allowbreak
	\varAccPoKGComm \| \allowbreak
	\varAccPoKHComm \| \allowbreak
	\varAccPoKGProof \| \allowbreak
	\varAccPoKHProof \| \allowbreak
	\mathfrak{t}_1 \| \allowbreak
	\mathfrak{t}_2 \| \allowbreak
	\mathfrak{t}_3 \| \allowbreak
	t_1 \| \allowbreak
	t_2 \| \allowbreak
	t_3 \| \allowbreak
	t_4
}
\newcommand{\expConcatAccPoKImproved}{
	\varAccPoKCoinComm \| \allowbreak
	\varAccPoKProofComm{c} \| \allowbreak
	\varAccPoKProofComm{w} \| \allowbreak
	\varAccPoKProofComm{r} \| \allowbreak
	\varAccPoKGComm \| \allowbreak
	\varAccPoKHComm \| \allowbreak
	\varAccPoKGProof \| \allowbreak
	\varAccPoKHProof \| \allowbreak
	\mathfrak{t}'_1 \| \allowbreak
	\mathfrak{t}'_2 \| \allowbreak
	\mathfrak{t}'_3 \| \allowbreak
	t'_1 \| \allowbreak
	t'_2 \| \allowbreak
	t'_3 \| \allowbreak
	t'_4
}
