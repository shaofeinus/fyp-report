\chapter*{Introduction}
Privacy is a major theme in any financial transaction system. While traditional banking systems offer privacy by keeping transaction records private, the centralised records can be exposed either maliciously or under the order of authorities. Since banks keep the real identities of transaction owners for accountability, privacy is completely destroyed when the records are exposed. Bitcoin as a digital currency system addresses some of the privacy problems of traditional banking by providing a level of “pseudonymity” through the use of arbitrary public key addresses as identities during payments (\S\ref{sec:1-Pseudonymity of Bitcoin}). This has partly driven the immense popularity of Bitcoin, which has a market capitalisation of USD\$10 billion and daily transactions volume of USD\$50 million as of October 2016\footnote{Based on statistics from https://coinmarketcap.com/}.

However due to the decentralised nature of Bitcoin, Bitcoin transaction records are unencrypted and publicly available. Even though transactions are executed with cryptic public key addresses, there are many techniques to de-anonymise Bitcoin transactions (\S\ref{sec:1-Ways to De-anonymise Bitcoin}). In order to preserve the privacy of Bitcoin, different innovations has been developed. Methods such Mixing can be easily integrated with Bitcoin, while others require separate digital currency systems, referred to as Altcoins, to be implemented (\S\ref{ch:Methods to Improve Privacy in Digital Currencies}).

One Altcoin, Zerocoin, which makes use of Non-interactive Zero Knowledge (NIZK) proofs to create and validate anonymous transactions, is identified as the subject of research due to its theoretical soundness and practical potential (\S\ref{sec:2-Selection Zerocoin as the Subject of Research}). This research attempts to address the computational and storage inefficiencies in Zerocoin (\S\ref{sec:2-Zerocoin Drawbacks}) so that it can be a viable substitute to the current Bitcoin system. This will be mainly done by modifying the NIZK proofs algorithms and related data structures (\S\ref{sec:3-NIZK Proofs in Zerocoin}) in the Zerocoin protocol. The proposed improvements will be benchmarked against the original Zerocoin protocol in terms of privacy preservation and computational performance (\S\ref{sec:4-Performance Metrics and Benchmarks}). The objective of the research is to achieve a considerable improvement in computational performance without significant compromise in privacy preservation.

At this point, this research has examined the theoretical basis and the construction of Zerocoin (\S\ref{ch:Analysis of Zerocoin}). The source code of the Zerocoin protocol developed by its creators has also been analysed and some areas for improvement have been identified (\S\ref{sec:3-Performance of Zerocoin and Areas for Improvement}). Improvements have also been proposed and tested by modifying the Zerocoin source code (\S\ref{ch:Contribution 1: A more Efficient Serial Number Signature of Knowledge} and \S\ref{ch:Contribution 2: A Smaller Accumulator Proof of Knowledge}). This research plans to continue refining the proposed improvements and test their performance on a network level by simulating a peer-to-peer Zerocoin network (\S\ref{sec:4-Performance Metrics and Benchmarks}). 