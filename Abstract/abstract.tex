% ************************** Thesis Abstract *****************************
% Use `abstract' as an option in the document class to print only the titlepage and the abstract.
\begin{abstract}
	Digital currencies such as Bitcoin provide some level of privacy because users can use multiple arbitrary identities to make transactions. However the caveat is that transaction records are public and unencrypted, and thus there are various ways to de-anonymise users. This research uses Bitcoin as a starting point to analyse the issues surrounding privacy in digital currencies and explores the various ways to overcome these issues. Zerocoin, an anonymous digital currency protocol based on Non-Interactive Zero-knowledge proofs, shows promise and is examined in detail. The main performance drawbacks in Zerocoin are the large computational and storage requirements for its proofs. If these problems are addressed, Zerocoin can be a commercially viable anonymous digital currency. This research attempts to reduce the performance overheads in Zerocoin by modifying the Accumulator Proof of Knowledge (AccPoK) and Serial Number Signature of Knowledge (SNSoK) in the protocol. The proposed SNSoK is a modification of the proof for a committed value in a Pedersen commitment, and is smaller and faster that the original SNSoK by about 40 times and 80 times respectively. The proposed AccPoK employs an optimisation technique that is used in the original SNSoK, but its performance has not been tested. Going forward, this research plans to continue enhancing the proposed improvements and simulate a Zerocoin network to test them on a network level. If time permits, this research will also explore ways to enhance the functionality aspects of Zerocoin.
  
\end{abstract}
